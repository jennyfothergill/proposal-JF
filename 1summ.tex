%%%%%%%%% PROJECT SUMMARY -- 1 page, third person
% e.g:  "The PI will prove" not "I will prove"

%Below are the pagination, font size, spacing and margin
%instructions for NSF proposals: \\
%
%FastLane does not automatically paginate a proposal.
%Each section of the proposal must be individually
%paginated prior to upload to the system. \\
%
%Use Computer Modern family of fonts at a font size of 11 points or
%larger. A font size of less than 10 points may be used for mathematical
%formulas or equations, figure, table or diagram captions and when
%using a Symbol font to insert Greek letters or special characters.
%The text must still be readable. The use of small type not in compliance with the NSF guidelines
%may be grounds for NSF to return the proposal without review. \\
%
%No more than 6 lines of text within a vertical space of 1 inch. \\
%
%Margins, in all directions, must be at least an inch. \\
%
%

\begin{center}
    \large{\textbf{Advancing organic photovoltaics with infrastructure for transferable, reproducible, useable, and extensible (TRUE) multiscale models}}
%\large{\textbf{ADVANCEMENT OF ORGANIC PHOTOVOLTAICS THROUGH ARTIFICIAL NEURAL NETWORKS}}
\end{center}
\begin{center}
    Jenny W. Fothergill
\end{center}
\begin{center}
    \textbf{Project Summary}
\end{center}
%\begin{center}
%\emph{Maximum of 1 page}
%\end{center}
% This should be a brief statement of the problem you plan to address.
% It should look something like an abstract. 

%The project summary should be a description of the proposed activity suitable
%for publication, no more than one page in length. It should not be
%an abstract of the proposal, but rather a self-contained description of
%the activity that would result if the proposal were funded. The summary
%should be written in the third person and include a statement of objectives
%and methods to be employed. It should be informative to other persons
%working in the same or related fields and understandable to a scientifically
%or technically literate lay reader. \\
%
%The summary must clearly address in separate statements (within the one-page summary):
%the intellectual merit of the proposed activity; and the broader impacts
%resulting from the proposed activity. Proposals that do not separately
%address both criteria within the one-page Project Summary will be returned without
%review. \\

Already the effects of human-influenced climate change are readily apparent and devastating. %%<hurtng everyone in particular poor and disadvantaged groups>.
Burning fossil fuels for energy is the main contributor to rising $CO_{2}$ levels, and the energy demand is only increasing.
This increase in energy demand along with the need to protect our climate makes finding efficient, renewable energy technologies of the utmost importance.
Harvesting the abundant solar energy is a good solution because it is renewable, non-polluting, and sustainable.
Devices which convert light to electricity, called photovoltaics, were originally made of silicon and required rare elements and careful, clean-room processing.
More recently, organic photovoltaic (OPV) devices have recieved attention for their simple, inexpensive manufacture.
Generally OPV devices are made of a multicompound mixture: usually an electron donor and acceptor which form a mosaic-like structure called a bulk-heterojunction (BHJ).
The compounds and polymers used in OPVs are made of common elements and can be synthesized with relative ease.
And the OPV BHJ can often be assembled using simple techniques like roll to roll printing in ambient conditions.
The process for OPV to generate electricity is as follows: 
Sunlight excites an electron, which forms an exciton.
This exciton travels to an acceptor/donor interface and is dissociated into free electron and hole.
Then the free charges travel to an electrode which generates current.
However, in order for OPV devices to be a viable solution, they need to be more efficient.
The efficiency of an OPV device can be improved by changing the chemistry of the electron acceptor and donor compounds, the polymer properties, and tuning the processing conditions of the BHJ.
With so many parameters, quickly this becomes an intractable problem for experiments to solve.
This is where molecular simulation comes in.

Molecular dynamics (MD), a simulation technique where atoms are represented by classical forces and their interactions are specified by attractive/repulsive potentials, can be used to probe the thermodynamic processes which determine how OPV compounds will self-assemble into a BHJ morphology.
Then we can use kinetic monte carlo (KMC) in concert with quantum chemical calculations (QCCs) to determine the charge mobility through these simulated morphologies.
By understanding how charges move through these systems, one can predict which OPVs will be most efficient.
However, this simulation workflow suffers from a problem of length scales:
Probing charge transport properties with QCCs requires the calculation of electron densities around each atom, which requires knowing the positions of each atom.
But equilibrating this morphology at volumes which exceed the exciton dissociation length (~5-10 nm) is prohibitive if every individual atom is considered.

Coarse grain (CG) models, which group multiple atoms together into "super atoms" and use an effective potential to specify the forces acting upon the site, can be used to access length and time scales more relevant to the bulk polymer morphology.
After a bulk morphology is obtained, we can use backmapping, a process in which the atomistic detail can be reintegrated into a coarse grain model, to restore atomistic detail for use in simulations like KMC/QCC.
This multiscale simulation workflow allows for OPV morphologies to be equilibrated at length scales relevant to the bulk polymer and to regain the atomistic detail more efficiently using a better starting structure.
Currently software packages exist that do this but they require tedious manual input and are highly susceptible to user error.
PI Fothergill will design a simple, chemically sensible coarse graining and backmapping package based on the SMILES chemical input language with a focus on OPVs.
This will allow for easier setup and execution of multiscale simulation, which can be part of the process to finding the next great OPV technology.

\begin{center}
    \textbf{Intellectual Merit}
\end{center}
% This is why your project is interesting and will help further
% knowledge in the field of science. 
%How important is the proposed activity to advancing
%knowledge and understanding within its own field or across different fields?
%How well qualified is the proposer (individual or team) to conduct the project?
%(If appropriate, the reviewer will comment on the quality of prior work.)
%To what extent does the proposed activity suggest and explore creative, original,
%or potentially transformative concepts? How well conceived and organized is the
%proposed activity? Is there sufficient access to resources?  \\

The proposed work is original, taking a systems view of software interoperability needed to improve organic photovoltaics with molecular simulation.
Through focus on transferable, reproducible, usable, and extensible simulation infrastructure, the proposed work is positioned to transform both the manufacturing of sustainable power technologies and the norms of practice in computational materials science.
The intellectual merit of project components---coarse-graining pipelines and multiscale simulations---includes the potential to transform materials simulations adjacent to organic photovoltaics, including polymers for structural composites and molecules of biomedical interest.
The specific insight into structure-property relationships in OPV materials also has intellectual transferability to broader polymer assembly problems, and charge transport problems of interest in piezoelectric materials, induction welding of polymers, and organic electronics broadly.
PI Fothergill is positioned to be successful in the proposed work, as a trained Software Carpentry instructor and open-source project contributor with published expertise in quantum chemical methods applied to charge transport in organic dyes, organometallic crystals, and coarse-grained materials simulation.
Ample computational resources are available to complete the proposed work, in the form of local HPC resources (Fry: 32 NVIDIA Pascal P100 gpus, R2: 28 P100 GPUs, Borah: ~30 GPUs coming online this year), Idaho resources (Idaho National Lab facilities: 100's of V100 GPUs) and national resources (125,000 GPU-hour allocation on XSEDE Bridges ).

\begin{center}
    \textbf{Broader Impacts}
\end{center}
% There are 4 kinds of broader impacts.
% 1. advance discovery and understanding while promoting teaching,
% training and learning
% 2. broaden the participation of underrepresented groups
% 3. disseminated broadly to enhance scientific and technological
% understanding
% 4. benefits of the proposed activity to society
%OPV devices reaching consistent efficacies of 15\% have the potential to create a global energy revolution.
%At this efficiency, OPV devices will surpass inorganic solar cells, and even gas and coal in terms of cost per watt of electricity generation.
%Thus, regardless of the current administration's reluctance to accept the reality of climate change, OPV devices will be seen superior to 'dirty' power production methods due to their competitive price.

The proposed work directly advances discovery and understanding of organic photovoltaic materials physics, molecular simulation, and scientific software development, while promoting teaching: Applying Software Carpentry pedagogical methods to tutorial development and TRUE simulations is the mechanism for simultanesouly advancing discovery while promoting teaching.
A direct consequence of applying The Carpentries' inclusivity-based pedagogical practices is that this work helps broaden participation of underrepresented groups: My own trajectory as a female in a STEM PhD program is empowered by this approach.
The products of this work---journal publications, data sets,  and open source software---are to be dissemenated free of subscription costs to users online.
By disseminating work broadly in this way, this work is positioned to have both broad impact in the application of the transferable tools we develop, and in the opportunity to influence the norms of scientific software development.
Finally, while undoubtebly aspirational, the proposed work is positioned to identify formulations of organic photovoltaics that can benefit society by significantly ameliorating global climate change.


