\documentclass{article}
\begin{document}
\begin{center}
    \large{\textbf{Advancing organic photovoltaics with reproducible, transferable coarse-grained models}}
\end{center}
\begin{center}
    Jenny W. Fothergill
\end{center}

Consumption of fossil fuels for energy is the main contributor to rising atmospheric $CO_{2}$. 
Solar power provides a non-polluting, sustainable option.
Organic photovoltaic (OPV) devices, which can convert incedent sunlight to electricity, are the focus of this research because they represent the best opportunity for cost-effective solar power.
The active layer of an OPV consists of an interpenetrating network of electron donor and acceptor compounds.
The efficiency of the device depends on the morphology of the active layer.

In this research we will use molecular simulation to predict the active layer morphology and, in turn, the device efficiency.
A key challenge herein is spanning the length and time scales relevant to active layer morphology kinetics and charge transport processes, demanding use of tools including density functional theory all the way up to coarse-grained models.
We propose developing new simulation tools that solve longstanding problems with simulation management, transferability, reproducibility, usability, and extensibility
Specifically, we develop tools in the Molecular Simulation Design Framework (MoSDeF) for automating the coarse-graining of organic photovoltaic chemistries, and predicting and analyzing equilibrated OPV morphologies.


\end{document}
