\documentclass{article}
\begin{document}
\begin{center}
    \large{\textbf{Advancing organic photovoltaics with infrastructure for transferable, reproducible, useable, and extensible (TRUE) multiscale models}}
\end{center}
\begin{center}
    Jenny W. Fothergill
\end{center}

Consumption of fossil fuels for energy is the main contributor to rising atmospheric $CO_{2}$. 
Solar power provides a non-polluting, sustainable alternative.
Organic photovoltaic (OPV) devices, which can convert incedent sunlight to electricity, are the focus of this research because they represent the best opportunity for cost-effective solar power.
OPV device efficiency depends on how the active layer morphology facilitates processes including light absorption and charge transport.

In this research we will use molecular simulations to predict the active layer morphologies for combinations of chemistries and how they contribute to device efficiency.
Connecting electronic processes like charge transport with larger length scales of active layer morphology demands simulation workflows between tools which has exposed significant interoperability challenges.
We propose developing new simulation infrastructure that solves longstanding problems with simulation management, transferability, reproducibility, usability, and extensibility.
Specifically, we develop tools in the Molecular Simulation Design Framework (MoSDeF) for automating the coarse-graining of OPV chemistries and predicting and analyzing equilibrated OPV morphologies.

\end{document}
