\documentclass{article}
\begin{document}
\begin{center}
    \large{\textbf{Simple, flexible generation of coarse grain models for novel organic photovoltaics}}
\end{center}
\begin{center}
    Jenny W. Fothergill
\end{center}
\begin{center}
    \textbf{Project Summary}
\end{center}

Consumption of fossil fuels for energy is the main contributor to rising atmospheric $CO_{2}$. 
Solar power provides a renewable, non-polluting, sustainable option.
Organic photovoltaic (OPV) devices, which can convert incedent sunlight to electricity, are the focus of this research because they represent the best opportunity for cost-effective solar power.
The active layer of an OPV consists of an interpenetrating network of electron donor and acceptor compounds.
The efficiency of the device depends on the morphology of the active layer.
In this research we will use molecular simulation to predict the active layer morphology and, in turn, the device efficiency.

Molecular dynamics (MD), a simulation technique where atoms are represented by classical forces and their interactions are specified by attractive/repulsive potentials, can be used to probe the thermodynamics of OPV self assembly.
The charge mobility through these morphologies can be calculated using kinetic monte carlo (KMC) in concert with quantum chemical calculations (QCCs).
By understanding how charges move through these systems, one can predict which OPVs will be most efficient.
However, this simulation workflow suffers from a problem of length scales:
Probing charge transport properties with QCCs requires the calculation of electron densities around each atom, which requires knowing the positions of each atom.
But equilibrating this morphology at volumes which exceed the exciton dissociation length (~5-10 nm) is prohibitive if every individual atom is considered.

Coarse grain (CG) models, which group multiple atoms together into ``super atoms'' with the net force on these atoms represented an effective potential, can be used to access length and time scales more relevant to the bulk polymer morphology.
After a bulk morphology is obtained, we can use backmapping, a process in which the atomistic detail can be reintegrated into a coarse grain model, to restore atomistic detail for use in simulations like KMC/QCC.
This multiscale simulation workflow allows for OPV morphologies to be equilibrated at length scales relevant to the bulk polymer and to regain the atomistic detail more efficiently using a better starting structure.
PI Fothergill will design a simple, chemically sensible coarse graining and backmapping package based on the SMILES chemical input language with a focus on OPVs.
This will allow for easier setup and execution of multiscale simulation, which can be part of the process to finding the next great OPV technology.
\end{document}
