\documentclass{article}
\title{Reading Notes}

\begin{document}
\title{}

today I got this repo set up. Next time, I'll reskim these papers:

MARTINI\cite{Marrink2007}

VOTCA\cite{Ruhle2009}

TraPPE\cite{Maerzke2011}

and read this: \cite{Chakraborty2018a} 

got busy with fancy-ing up the GIXStapose repo, helping rachel, office hours, and atom-typing with gaff-foyer with ryan and chris. Briefly skimmed the chakraborty paper-- they consider all possible mappings that a cg compound can have in order to systematically pick a cging scheme.
OK Chakraborty notes:
-exclude cg mapping schemes to those that maintain symmetry
-Zhang and Voth designed CG mapping operators by preserving "essential dynamics"
-Cao and Voth worked on centering functions COM vs center of charge. They found center of change was better for preserving intermolecular structure.
-Wagner generalize centering functions to arbitrary thermodynamic quantities
-Marrink grouped four heavy atoms and showed that this simple consistent scheme works.
CG models are validated by preservation of rdfs and velocity autocorrelation functions (vacfs)
mapping operator tree

Today I'm reading the VOTCA\cite{Ruhle2009} paper with the goal of answering these questions:
how do they choose a chromophore?

how do they calculate the transfer integral?

how do they do backmapping?

other notes:
CG methods--IBI, force matching, inverse MC. This paper goes over details of these and how they're implemented in votca. This wasn't really what I wanted to read about so I skimmed.
They use ESPResSo++ for quantum calculations
OK, this isn't the right paper to read about CT stuff or backmapping. Let's look at some of these [http://www.votca.org/citations/using-votca](http://www.votca.org/citations/using-votca)
Also I'm noticing in this list of papers--Marrink (cited by chakraborty) is here and there is a paper mentioning the importance of side-chains for charge-mobility! \cite{May2011}
Next time read about tie-chains \cite{May2011} and/or charge mobility \cite{Lukyanov2010}
Lukyanov--sounds a lot like morphct, gaussian density of states? I thought their (votca's) method was like morphct but now I'm not sure... It seems like they're using a simplification and not doing dft.

https://pubs.acs.org/doi/10.1021/ja104380c - 
rubrene record charge carrier 15 cm^2/(V s) single crystal by vapor deposition. typical for thin-film is 1 cm^2/(V s)
good discussion of some of the assumption of marcus theory. 
They start with single crystal structures from xray data. They calculate reorganization energies with b3lyp 6-311G(d,p) and transfer integrals with ZINDO. (check their SI)


\bibliography{library}
\bibliographystyle{ieeetr}
\end{document}

