\documentclass{article}
\title{Reading Notes}

\begin{document}
\title{}

today I got this repo set up. Next time, I'll reskim these papers:

MARTINI\cite{Marrink2007}

VOTCA\cite{Ruhle2009}

TraPPE\cite{Maerzke2011}

and read this: \cite{Chakraborty2018a} 

got busy with fancy-ing up the GIXStapose repo, helping rachel, office hours, and atom-typing with gaff-foyer with ryan and chris. Briefly skimmed the chakraborty paper-- they consider all possible mappings that a cg compound can have in order to systematically pick a cging scheme.
OK Chakraborty notes:
-exclude cg mapping schemes to those that maintain symmetry
-Zhang and Voth designed CG mapping operators by preserving "essential dynamics"
-Cao and Voth worked on centering functions COM vs center of charge. They found center of change was better for preserving intermolecular structure.
-Wagner generalize centering functions to arbitrary thermodynamic quantities
-Marrink grouped four heavy atoms and showed that this simple consistent scheme works.
CG models are validated by preservation of rdfs and velocity autocorrelation functions (vacfs)
mapping operator tree

Today I'm reading the VOTCA\cite{Ruhle2009} paper with the goal of answering these questions:
how do they choose a chromophore?

how do they calculate the transfer integral?

how do they do backmapping?

other notes:
CG methods--IBI, force matching, inverse MC. This paper goes over details of these and how they're implemented in votca. This wasn't really what I wanted to read about so I skimmed.
They use ESPResSo++ for quantum calculations
OK, this isn't the right paper to read about CT stuff or backmapping. Let's look at some of these [http://www.votca.org/citations/using-votca](http://www.votca.org/citations/using-votca)
Also I'm noticing in this list of papers--Marrink (cited by chakraborty) is here and there is a paper mentioning the importance of side-chains for charge-mobility! \cite{May2011}
Next time read about tie-chains \cite{May2011} and/or charge mobility \cite{Lukyanov2010}
Lukyanov--sounds a lot like morphct, gaussian density of states? I thought their (votca's) method was like morphct but now I'm not sure... It seems like they're using a simplification and not doing dft.

\cite{Vehoff2010a}
rubrene record charge carrier $15 \frac{cm^{2}}{V s}$ single crystal by vapor deposition. typical for thin-film is $1 \frac{cm^{2}}{V s}$
good discussion of some of the assumption of marcus theory. 
Drude model (electrons bounce off stationary nuclei like a pinball machine) works for perfect crystals at low temp.
At RT it doesn't work. Marcus theory assumes nuclear dynamics much slower (Borne-Oppenheimer) and electronic coupling is weak (what does this mean exactly?). Static disorder is based on the electronic DOS and the hopping rate between localized states. (As opposed to what?)
They start with single crystal structures from xray data. They calculate reorganization energies with b3lyp 6-311G(d,p) and transfer integrals with ZINDO. (check their SI)
How are they able to calculate direction-resolved transfer integrals?? very cool. What exactly do the transfer integral graphs mean? (Figure 2) Had a helpful chat with Chris about this and it helped me to think through the figures. The y-axis on the transfer integral is counts and the x-axis is energies--it makes sense that movement of an electron in different directions would require different energies and/or be more likely. Still not sure how to calculate it though. 

Today I'm digging around in more of the votca stuff to try and figure out how to run kmc there. They use ORCA too! Also their tutorials and documentation REALLY SUCK. [https://github.com/votca/xtp-tutorials](https://github.com/votca/xtp-tutorials) in particular I'm looking at LAMMPS/KMC\_Thiophene.
OK. either I am dumb or that tutorial sucks. I can tell that they're using orca and the inputs/oputputs the use but there's no explanation of what they're doing. It looks like they calculate the reorganization energy, but they don't define a chromophore--which makes sense for thiophene but what about when your molecule is ginormous?
\cite{Ruhle2011b} discusses charge transport in amorphous films of [tris-(8-hydroxyquinoline)aluminum](https://en.wikipedia.org/wiki/Tris(8-hydroxyquinolinato)aluminium). Oh Figure 1 is nice--it is a workflow of what can be done in votca.
11A talks about how to parameterize new contributions to a force-field using first-principles.
11B choosing a "chromophore"-- rigid fragments vs conjugated segments. they broke apart the chromophore in a polymer like p3ht using the dihedral angle--although they validated this choice with other methods. Interesting-- they replace the chromophores with those which have been geometry optimized using DFT. then they make a neighbor list
I'll come back to this paper later. I at least was able to answer some of my questions about choosing a chromophore.
11C talks about calculating the transfer integral for each pair and why ZINDO works and when it breaks down.

rereading the perspective paper \cite{Jankowski2019} looking for things to read next:
Braun 2018 foundations in molecular simulation
https://doi.org/10.33011/ livecoms.1.1.5957
74-89 are relevant to coarse-graining
92-95 start here for opvs

Today I'm thinking about some of the big picture stuff related to my proposal:
Science-wise I am thinking of my proposal like this: Humans use energy and burning fossil fuels to get that energy really sucks. rising CO2 concentration in the atmosphere contributes to global warming. Sustainable, low-carbon footprint energy generation is vital to protect the precious climate which sustains all life. Energy generation using OPV devices is one form of alternative energy. But there are a lot of potential OPV molecules out there! So how can we develop inexpensive, fast methods for screening this ZOO of compounds? Well we can use modelling techniques and a computer. Computational simulations of organic photovoltaics suffer from a problem of length scales: OPV morphologies can have bulk features on the scale of 10s of nanometers (BHJ, exciton dissociation length, pi-stacking) which affect the morphology, yet still we are interested in the electronic properties of this system--a fundementally quantum mechanical property--taking place on the angstrom scale where with current computers we can only realistically look at up to 100's of atoms. <so we break apart the problem MD to get a morphology, Zindo to calculate energy splitting of neighbors-->transfer integral, KMC to estimate CT, talk about results of evan/matty/votca> What am I going to add? more robust/automated coarse-graining and backmapping, calculation of reorganization energy, chromophore detection.
Things I care about: The environment--I like working on a project that theoretically could help find the next molecule for OPV technology. Sometimes it seems very far removed, but having sustainable technology as an ultimate goal matters to me. I also care a lot about help make science more accessible to everyone: I find myself really excited about creating tools which help to visualize the unseeable and to communicate complex ideas in a simple understandable way. I'm not sure this has anything to do with anything, but it's definitely a developing passion.

I'm also reading this paper today https://arxiv.org/pdf/2003.02031.pdf which is a bout TRUE simulations.
notes:
GUI's hurt reproducibility, MS Excel hides the formula's and order of calculation (p 4)
poorly documented code and raw data files dont help reproducibility (looking at you votca) (p4)
*I want to include start to finish notebooks with any paper I publish...* Also docker containers with my exact software stack.
I love the examples in this paper :) It feels a lot like our perspective paper! I'm skimming them but I love that notebooks are included in the SI.
They qualify the GUI's are bad statement saying that GUIs are only bad *if they hide the workflow details.* 


\bibliography{library}
\bibliographystyle{ieeetr}
\end{document}

